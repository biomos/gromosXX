\documentclass[12pt,a4paper,english]{article} 

\voffset -1.5cm
\textheight 24cm
\hoffset -1.2cm
\textwidth 15.6cm

%%\usepackage[T1]{fontenc}
\usepackage[latin1]{inputenc} 
\usepackage{babel} 
\usepackage{setspace}

%% \usepackage{fancyheadings}
\usepackage{pslatex}
\usepackage{epsfig}
\usepackage{graphicx}
\usepackage{units}
\usepackage{booktabs}

%%\usepackage{cite}
\usepackage{citesort}  
\usepackage{overcite}
%%\usepackage{latexsym}

\usepackage{listings}

%%\doublespacing

\newcommand{\mbf}[1]{\mathbf{#1}}
\newcommand{\deri}[2]{\frac{\partial #2}{\partial #1}}
\renewcommand{\baselinestretch}{2.0}
\newcommand{\noun}[1]{\textsc{#1}}

\title{GromosXX}
\date{\today}
\author{Markus Christen${}^{\dagger}$
\thanks{Corresponding author (markus@igc.phys.chem.ethz.ch)}}

\begin{document}

\lstset{language=C++}

\maketitle
\noindent ${}^{\dagger}${Laboratory of Physical Chemistry, Swiss Federal Institute of
Technology Z\"urich, ETH-H\"onggerberg, CH-8093 Z\"urich, Switzerland}

\newpage

\begin{abstract}
We present the newest version of the Groningen Molecular Simulation
program package, GROMOS05. It has been developed for the dynamic
modelling of (bio)molecules using the methods of molecular dynamics,
stochastic dynamics, and energy minimization. An overview of GROMOS05 is
given, highlighting features not present in the last major release,
GROMOS96. The organization of the program package is outlined and the
included analysis package Gromos++ is described. Finally, three
applications illustrating the functionality are presented: gug, bla, and dong.
\newline
{\textbf{Keywords:} molecular dynamics simulation, programming, gromos
}
\end{abstract}

\newpage

\section{Introduction}
Starting with GROMOS87 a program package to facilitate research efforts
in the field of biomolecular simulation in a university environment has
been developed. After the last major release, GROMOS96, GROMOS has found
widespread use (hundreds of licences in over 40 countries on all
continents), helped by the fact that it has been designed for eaese of
extendability and that the complete source code is made available to
research establishments for a nominal fee. The program code has been
further developed in the group for computational chemistry at ETH
Z\"urich (Switzerland) throughout the recent years, leading now to a new
major release, GROMOS05. The enhancements were governed by following
criteria: (1) interest of our research group, (2) ease of use, (3)
extendability, (4) demonstrated usefulness or efficiency, (5)
well-defined and correct formulae and algorithms, and (6) computational
efficiency. The second criterium lead to a complete rewrite of the
analysis tools, now contained in the Gromos++ analysis package and the
third criterium lead to a complete rewrite of the MD engine into C++
while the reliable Fortran MD engine (promd) with many new features
features is still present in the current release.

In the next section, features of Gromos05 are described, highlighting
new functionality. Then the organization of the source
code is discussed. Section 4 gives an overview over the programs present
in the Gromos++ analysis package and finally in section 5 examples of
applications are shown.

\section{Features}

Here, the features of the MD engines will be listed. The enhanced
fortran MD engine (promd) retains all features of GROMOS96 and adds new
functionality. The C++ version contains most of the GROMOS96 features
(except 4D and path integral simulation) and a subset of the new
functionality. 

Features included are (non-complete):
\begin{itemize}
  \item Molecular Dynamics, Stochastic Dynamics simulation and energy
  \item Periodic boundary conditions (vacuum, rectangular, truncated
  octahedron, monoclinic and triclinic)
  minimiaation (steepest descent and conjugate gradient)
  \item Temperature control (weak coupling, Nose-Hoover, Nose-Hoover
  chains)
  \item Pressure control (weak coupling: isotropic, anisotropic and full
    anisotropic), separate coupling of any group of degrees of freedom
    for internal / rotational or translational temperature baths is possible
  \item Longrange interactions: straight cutoff, reaction field, ewald,
    particle-mesh-ewald, particle-partice-partice-mesh 
  \item Chargegroup based or atomic cutoff for nonbonded interactions
  \item Non-physical interaction: distance, dihedral, NOE and J-value
    restraints; distance constraints (SHAKE, M-SHAKE, LINCS)
  \item Relative free energy calculation, possibly including soft-core
    nonbonded interactions
  \item Enhanced sampling (Local elevation, replica exchange)
  \item Path integral simulation
  \item Grid based pairlist construction
  \item MPI and OMP parallelization
\end{itemize}

A complete description of the available features is included within the
manual.

\section{Implementation}

\subsection{promd (Fortran MD engine)}

The fortran MD engine promd is an enhancement of the GROMOS96 md
engine. It is written in Fortran77 with the exception of the use of
include files. An option to select different nonbonded routines has been
added. Macros to get rid of unwanted features (like 4D simulation or
unused periodicity code) have been added. Using these it is possible to
get a faster specialized build of promd. A cache-effective pairlist
generation method has been added. It additionally uses OpenMP to run in
parallel on shared memory multiprocessor systems.

\subsection{xxmd (C++ MD engine)}

The C++ MD engine (xxmd) has been rewritten from scratch. The major
motivation was to further increase the modularity and therefore the
extendability of the MD program. The code is split into two parts, the
first being an MD library containing basic functions necessary to run an
MD simulation, the second part the actual MD program. This second part
is very small. It is quite easy to add other specialized MD programs,
that make use of a subset of the functions provided in the library or
apply them in a different order. The source code of the library is again
split up into nine different parts: math, simulation, topology,
configuration, algorithm, interaction, io, util and check.

\begin{itemize}
  \item math contains classes for vectors, matrices and vector arrays,
    mathematical operations, physical constants and periodic boundary
    treatment.
  \item simulation contains the simulation parameters supplied to run
    an MD simulation.
  \item topology contains the topology of the simulated system, also
    includes possibly a perturbation topology.
  \item configuration contains the state of a system: its coordinates,
    velocities, forces, restraints data and so on.
  \item algorithm contains classes that use information from simulation
    and topology to act upon a configuration. All steps during an MD
    simulation can be described using an algorithm.
  \item interaction contains the biggest algorithm: the
    forcefield. Here, all interaction terms and their parameters are
    defined.
  \item io contains classes to read in or write out information. All
    file access is block oriented and human readable.
  \item util contains a few extra classes that are necessary to setup an
    MD simulation but don't belong exactly to it. Parsing command line
    arguments, generation of initial velocities or setting debug levels
    are examples of classes found herein.
  \item check contains test routines. Testing includes automatic calculation of
    energies under different conditions, then calculation of forces,
    energy lambda derivatives and virial and comparing to values
    obtained by finite difference calculations.
\end{itemize}

One step of an MD simulation consists of several algorithms applied to
the configuration in the right order. The Algorithm_Sequence class is a
container for all the algorithms, they have to be inserted in the
correct order. Before the start of a simulation, all algorithms will be
initialized (by calling the init() function), during an MD step, the
algorithms are applied (apply()). The forcefield is also an algorithm,
which, when applied, calculates the forces and interaction energies of
all forcefield terms for the complete system. The forcefield terms
itself are Interaction classes. The forcefield is therefore a container
to store the different interaction objects. When the forcefield gets
applied, it calls calculate_interactions() on all interaction
objects. These, in term, calculate the bonded interactions (bond length,
bond angles, improper dihedrals and dihedrals), the nonbonded
interactions (pairlist construction, long-range interactions,
short-range interactions and virial) and the non-physical forces
(distance-, dihedral-, NOE- or J-value-restraints).

\subsubsection{Parallelization}

\section{Analysis: Gromos++}

\section{Applications}

\section{Conclusions}

%\cite{Gunsteren:99.11}

\section{Acknowledgements}

Financial support by the National Center of Competence in Research (NCCR)
Structural Biology of the Swiss National Science Foundation (SNSF) is
gratefully acknowledged.

\begin{thebibliography}{10}

\bibitem{C++standard}
Programming languages - C++, ISO 14882, 2003.

\bibitem{AllenTildesley}
M. P. Allen and D. J. Tildesley, Computer Simulations of Liquids (Oxford
Science, Oxford, 1987).

\bibitem{Gunsteren:96.40}
  W.F. van Gunsteren, S.R. Billeter, A.A. Eising, P.H. H�nenberger, P. Kr�ger, 
  A.E. Mark, W.R.P. Scott, I.G. Tironi, Biomolecular Simulation: The GROMOS96
  Manual and User Guide, Vdf Hochschulverlag AG an der ETH Z�rich, Z�rich, Switzerland.

\bibitem{Gunsteren:99.11}
  W.R.P. Scott, P.H. H�nenberger, I.G. Tironi, A.E. Mark, S.R. Billeter,
  J. Fennen, A.E. Torda, T. Huber, P. Kr�ger and W.F. van Gunsteren, 
  J. Phys. Chem. A {\bf 103}, 3596 (1999).

\bibitem{Gunsteren:01.26}
  L.D. Schuler, X. Daura and W.F. van Gunsteren,
  J. Comput. Chem. {\bf 22}, 1205 (2001).

\end{thebibliography}

\pagebreak[4]

\renewcommand{\listtablename}{Table Captions}
\listoftables
\clearpage
\thispagestyle{empty}

\clearpage
\pagebreak[4] 

\renewcommand{\listfigurename}{Figure Captions}
\listoffigures

\clearpage
\pagebreak[4] 

\clearpage
\pagebreak[4] 

\clearpage
\end{document}
